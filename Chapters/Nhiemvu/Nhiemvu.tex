\begin{table}
\centering
\footnotesize
\begin{tabular}{p{0.5\linewidth} p{0.5\linewidth}}

\begin{center}
\makecell{TRƯỜNG ĐẠI HỌC SƯ PHẠM KỸ THUẬT \\ TP HCM}

KHOA CƠ KHÍ CHẾ TẠO MÁY
\end{center} & 
\begin{center}
CỘNG HOÀ XÃ HỘI CHỦ NGHĨA VIỆT NAM

\textit{Độc Lập - Tự Do - Hạnh Phúc}
\end{center} \\ 
\end{tabular} 
\end{table}
\begin{center}
\subsection*{NHIỆM VỤ ĐỒ ÁN MÔN HỌC HỆ THỐNG CƠ ĐIỆN TỬ}
\end{center}
\addcontentsline{toc}{section}{NHIỆM VỤ ĐỒ ÁN MÔN HỌC HỆ THỐNG CƠ ĐIỆN TỬ}

\textbf{Giảng viên hướng dẫn: } ThS Nguyễn Vũ Lân

\textbf{Sinh viên thực hiện: }
\begin{center}
\begin{tabular}{cc}
Võ Văn Đoàn & 17146256 \\ 

Trần Minh Toàn & 17146343 \\ 

Vũ Văn Phiêu & 17146306 \\ 
\end{tabular} 
\end{center}

\textbf{1. Tên đề tài: } Cân bằng cường độ sáng cho không gian phòng học.

\textbf{2. Các số liệu, tài liệu ban đầu: }

Kích thước mô hình phòng học: $ 1000 \times 600 \times 300 $ (Dài $\times$ rộng $\times$ cao).

\textbf{3. Nội dung chính của đồ án: }
\begin{itemize}
\item Thiết kế, thi công, lắp đặt mô hình phòng học.
\item Thi công, lắp đặt hệ thống động cơ, hệ thống đèn chiếu sáng.
\item Lắp đặt hệ thống cảm biến ánh sáng.
\item Viết chương trình điều khiển trên VĐK và phần mềm điều khiển trên PC.
\end{itemize}

\textbf{4. Các sản phẩm dự kiến: }
\begin{itemize}
\item Mô hình phòng học.
\item Hệ thống điều khiển.
\item Phần mềm điều khiển.
\end{itemize}

\textbf{5. Ngày giao đồ án: } 09/03/2021

\textbf{6. Ngày nộp đồ án: } 25/06/2021

\textbf{7. Ngôn ngữ trình bày: }
\begin{itemize}
\item Bản bảo cáo: Tiếng Việt.
\item Trình bày phản biện: Tiếng Việt.
\end{itemize}

\begin{center}
\begin{tabular}{w{c}{0.3\linewidth} w{c}{0.3\linewidth} w{c}{0.3\linewidth}}

\textbf{TRƯỞNG KHOA} & \textbf{TRƯỞNG BỘ MÔN} & \textbf{GVHD} \\ 

\textit{(Ký, ghi rõ họ tên)} & \textit{(Ký, ghi rõ họ tên)} & \textit{(Ký, ghi rõ họ tên)} \\ 

\end{tabular} 
\end{center}